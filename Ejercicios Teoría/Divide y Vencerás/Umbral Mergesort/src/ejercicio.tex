\documentclass[12pt,spanish]{article}
\usepackage[spanish]{babel}
\usepackage{graphicx}
\usepackage{multirow}
\usepackage[hidelinks]{hyperref}
\usepackage{color}
\usepackage{caption}
\usepackage{multirow}
\usepackage{multicol}
\usepackage[outputdir=build]{minted}
\usepackage{float}
\usepackage{array}
\graphicspath{ {../img/} {../../LaTeX/img/} {/home/csp98/latex/img/}}
\selectlanguage{spanish}
\usepackage[utf8]{inputenc}
\usepackage{graphicx}
\usepackage[a4paper,left=3cm,right=2cm,top=2.5cm,bottom=2.5cm]{geometry}
\newtheorem{ppio}{Principio }
\makeindex

\begin{document}
\begin{titlepage}

\newlength{\centeroffset}
\setlength{\centeroffset}{-0.5\oddsidemargin}
\addtolength{\centeroffset}{0.5\evensidemargin}
\thispagestyle{empty}

\noindent\hspace*{\centeroffset}
\begin{minipage}{\textwidth}

\centering
\includegraphics[width=0.9\textwidth]{logo_ugr.jpg}\\[1.4cm]

\textsc{ \Large Algorítmica\\[0.2cm]}
\textsc{GRADO EN INGENIERÍA INFORMÁTICA}\\[1cm]

{\Huge\bfseries Ejercicio de clase\\}
\noindent\rule[-1ex]{\textwidth}{3pt}\\[3.5ex]
{\large\bfseries Umbral de \textit{mergesort}}
\end{minipage}

\vspace{1.5cm}
\noindent\hspace*{\centeroffset}
\begin{minipage}{\textwidth}
\centering

\textbf{Autores}\\ {Carlos Sánchez Páez}\\[2.5ex]
\includegraphics[width=0.3\textwidth]{etsiit_logo.png}\\[0.1cm]
\vspace{1.5cm}
\includegraphics[width=0.5\textwidth]{decsai.jpg}\\[0.1cm]
\vspace{1cm}
\textsc{Escuela Técnica Superior de Ingenierías Informática y de Telecomunicación}\\
\vspace{1cm}
\textsc{Curso 2017-2018}
\end{minipage}
\end{titlepage}
\tableofcontents
\thispagestyle{empty}
\listoftables
\newpage
\setcounter{page}{1}

\section{Enunciado}
\large{Realizar un estudio empírico para determinar un buen umbral para el algoritmo de ordenación por mezcla o mergesort (que utiliza como algoritmo auxiliar el algoritmo de ordenación por inserción para casos de tamaño menor que el umbral).}

\section{Resolución}

\subsection{Metodología}
Para realizar el ejercicio, modificamos el código de \emph{mergesort.cpp} disponible en \emph{DECSAI} para que acepte el tamaño del umbral como parámetro de ejecución del programa.\\
La metodología será realizar 20 ejecuciones con un tamaño de vector constante (10.000.000) y un umbral variable para determinar el tamaño óptimo del límite.\\
Para elo, he desarrollado el siguiente \textit{script}:
\subsection{Script desarrollado}
\begin{minted}[linenos]{bash}
#!/bin/bash
if [ $# -eq 3 ]
then
	i="0"
	tam=$2
	#Primer argumento: programa a ejecutar
	#Segundo argumento: tamaño inicial del umbral
	#Tercer argumento : incremento
	while [ $i -lt 20 ]
	do
		./$1 $tam 1000000 >> ./$1.dat
		i=$[$i+1]
		tam=$[$tam+$3]
	done
else
	echo "Error de argumentos"
fi
\end{minted}
%$
\subsection{Datos de las ejecuciones}
Comenzamos ejecutando desde 10 hasta 100:
\begin{table}[H]
\centering
\begin{tabular}{|c|c|}
\hline
\textbf{Tamaño de umbral} & \textbf{Tiempo de ejecución (s)}\\
\hline
10 & 0.246989\\
\hline
20 & 0.242702\\
\hline
30 & 0.245441\\
\hline
40 & 0.249388\\
\hline
50 & 0.25211\\
\hline
60 & 0.251404\\
\hline
70 & 0.283634\\
\hline
80 & 0.292154\\
\hline
90 & 0.283183\\
\hline
100 & 0.281779\\
\hline
110 & 0.283369\\
\hline
120 & \textcolor{red}{0.28545}\\
\hline
130 & \textcolor{red}{0.362056}\\
\hline
140 & 0.36134\\
\hline
150 & 0.361897\\
\hline
160 & 0.362099\\
\hline
170 & 0.362484\\
\hline
180 & 0.362892\\
\hline
190 & 0.362747\\
\hline
200 & 0.36144\\
\hline
\end{tabular}
\caption{Primera aproximación}
\end{table}
Vemos que entre $t_{umbral}=120$ y $t_{umbral}=130$ se produce un gran salto. Por tanto, volvemos a lanzar el script para afinar a las unidades:
\begin{table}[H]
\centering
\begin{tabular}{|c|c|}
\hline
\textbf{Tamaño de umbral} & \textbf{Tiempo de ejecución (s)}\\
\hline
120 & 0.283637\\
\hline
121 & 0.285872\\
\hline
122 & \textcolor{red}{0.284597}\\
\hline
123 & \textcolor{red}{0.360809}\\
\hline
124 & 0.362452\\
\hline
125 & 0.360889\\
\hline
126 & 0.362977\\
\hline
127 & 0.361814\\
\hline
128 & 0.362678\\
\hline
129 & 0.361538\\
\hline
130 & 0.370429\\
\hline
\end{tabular}
\caption{Segunda aproximación}
\end{table}
\subsection{Conclusión}
Por tanto, podemos concluir que un umbral adecuado para el algoritmo \emph{mergesort} es \textbf{122}.


\section{Anexo: Código fuente utilizado}
\begin{minted}[linenos]{c++}

#include <iostream>
using namespace std;
#include <ctime>
#include <cstdlib>
#include <climits>
#include <cassert>

/**
   @brief Ordena un vector por el método de mezcla.

   @param T: vector de elementos. Debe tener num_elem elementos.
             Es MODIFICADO.
   @param num_elem: número de elementos. num_elem > 0.

   Cambia el orden de los elementos de T de forma que los dispone
   en sentido creciente de menor a mayor.
   Aplica el algoritmo de mezcla.
*/
inline static
void mergesort(int T[], int num_elem, const int UMBRAL_MS);

/**
   @brief Ordena parte de un vector por el método de mezcla.

   @param T: vector de elementos. Tiene un número de elementos
                   mayor o igual a final. Es MODIFICADO.
   @param inicial: Posición que marca el incio de la parte del
                   vector a ordenar.
   @param final: Posición detrás de la última de la parte del
                   vector a ordenar.
		   inicial < final.

   Cambia el orden de los elementos de T entre las posiciones
   inicial y final - 1 de forma que los dispone en sentido 
   creciente de menor a mayor.
   Aplica el algoritmo de la mezcla.
*/
static void mergesort_lims(int T[], int inicial, int final, 
const int UMBRAL_MS);

/**
   @brief Ordena un vector por el método de inserción.

   @param T: vector de elementos. Debe tener num_elem elementos.
             Es MODIFICADO.
   @param num_elem: número de elementos. num_elem > 0.

   Cambia el orden de los elementos de T de forma que los dispone
   en sentido creciente de menor a mayor.
   Aplica el algoritmo de inserción.
*/
inline static
void insercion(int T[], int num_elem);

/**
   @brief Ordena parte de un vector por el método de inserción.

   @param T: vector de elementos. Tiene un número de elementos
                   mayor o igual a final. Es MODIFICADO.
   @param inicial: Posición que marca el incio de la parte del
                   vector a ordenar.
   @param final: Posición detrás de la última de la parte del
                   vector a ordenar.
		   inicial < final.

   Cambia el orden de los elementos de T entre las posiciones
   inicial y final - 1 de forma que los dispone en sentido 
   creciente de menor a mayor.
   Aplica el algoritmo de la inserción.
*/
static void insercion_lims(int T[], int inicial, int final);

/**
   @brief Mezcla dos vectores ordenados sobre otro.

   @param T: vector de elementos. Tiene un número de elementos
                   mayor o igual a final. Es MODIFICADO.
   @param inicial: Posición que marca el incio de la parte del
                   vector a escribir.
   @param final: Posición detrás de la última de la parte del
                   vector a escribir
		   inicial < final.
   @param U: Vector con los elementos ordenados.
   @param V: Vector con los elementos ordenados.
             El número de elementos de U y V sumados debe 
             coincidir con final - inicial.

   En los elementos de T entre las posiciones inicial y final 
   - 1 pone ordenados en sentido creciente, de menor a mayor,
   los elementos de los vectores U y V.
*/
static void fusion(int T[], int inicial, int final, 
int U[], int V[]);

/**
   Implementación de las funciones
**/

inline static void insercion(int T[], int num_elem){
	insercion_lims(T, 0, num_elem);
}


static void insercion_lims(int T[], int inicial, int final){
	int i, j;
	int aux;
	for (i = inicial + 1; i < final; i++) {
		j = i;
		while ((T[j] < T[j - 1]) && (j > 0)) {
			aux = T[j];
			T[j] = T[j - 1];
			T[j - 1] = aux;
			j--;
		};
	};
}


void mergesort(int T[], int num_elem, const int UMBRAL_MS){
	mergesort_lims(T, 0, num_elem, UMBRAL_MS);
}

static void mergesort_lims(int T[], int inicial, 
int final, const int UMBRAL_MS{
	if (final - inicial < UMBRAL_MS)
		insercion_lims(T, inicial, final);
	else {
		int k = (final - inicial) / 2;

		int * U = new int [k - inicial + 1];
		assert(U);
		int l, l2;
		for (l = 0, l2 = inicial; l < k; l++, l2++)
			U[l] = T[l2];
		U[l] = INT_MAX;

		int * V = new int [final - k + 1];
		assert(V);
		for (l = 0, l2 = k; l < final - k; l++, l2++)
			V[l] = T[l2];
		V[l] = INT_MAX;

		mergesort_lims(U, 0, k, UMBRAL_MS);
		mergesort_lims(V, 0, final - k, UMBRAL_MS);
		fusion(T, inicial, final, U, V);
		delete [] U;
		delete [] V;
	};
}

static void fusion(int T[], int inicial, int final, int U[],
 int V[]){
	int j = 0;
	int k = 0;
	for (int i = inicial; i < final; i++){
		if (U[j] < V[k]) {
			T[i] = U[j];
			j++;
		} else {
			T[i] = V[k];
			k++;
		};
	};
}

int main(int argc, char * argv[])
{

	if (argc != 3){
		cerr << "Formato " << argv[0] << " <umbral> 
		<num_elem>" << endl;
		return -1;
	}
	const int UMBRAL_MS = atoi(argv[1]);
	int n = atoi(argv[2]);

	int * T = new int[n];
	assert(T);

	srandom(time(0));

	for (int i = 0; i < n; i++)
		T[i] = random();
	
	clock_t tantes;    
	clock_t tdespues; 
	tantes = clock();
	mergesort(T, n, UMBRAL_MS);
	tdespues = clock();
	cout << UMBRAL_MS << "\t" << 
	((double)(tdespues - tantes))
	 / CLOCKS_PER_SEC << endl;

	delete [] T;

	return 0;
};

\end{minted}

\end{document}
