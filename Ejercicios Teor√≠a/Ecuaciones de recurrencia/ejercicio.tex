\documentclass[12pt,spanish]{article}
\usepackage[spanish]{babel}
\usepackage{graphicx}
\usepackage{svg}

\usepackage{multirow}
\usepackage{xfrac}
\usepackage{amsmath}
\usepackage[hidelinks]{hyperref}
\usepackage{fixmath}
\usepackage{caption}
\usepackage{multicol}
\usepackage[skins,minted,breakable]{tcolorbox}
\usepackage{float}
\usepackage{array}
\graphicspath{ {./img/} {../../LaTeX/img/} {/home/csp98/latex/img/}}
\selectlanguage{spanish}
\usepackage[utf8]{inputenc}
\usepackage{graphicx}
\usepackage[a4paper,left=3cm,right=2cm,top=2.5cm,bottom=2.5cm]{geometry}
\newtheorem{ppio}{Principio }
\makeindex

\begin{document}
\begin{titlepage}

\newlength{\centeroffset}
\setlength{\centeroffset}{-0.5\oddsidemargin}
\addtolength{\centeroffset}{0.5\evensidemargin}
\thispagestyle{empty}

\noindent\hspace*{\centeroffset}
\begin{minipage}{\textwidth}

\centering
\includegraphics[width=0.9\textwidth]{logo_ugr.jpg}\\[1.4cm]

\textsc{ \Large Algorítmica\\[0.2cm]}
\textsc{GRADO EN INGENIERÍA INFORMÁTICA}\\[1cm]

{\Huge\bfseries Ejercicios ecuaciones de recurrencia\\}
\noindent\rule[-1ex]{\textwidth}{3pt}\\[3.5ex]
{\large\bfseries Método de expansión}
\end{minipage}

\vspace{1.5cm}
\noindent\hspace*{\centeroffset}
\begin{minipage}{\textwidth}
\centering

\textbf{Autor}\\ {Carlos Sánchez Páez}\\[2.5ex]
\includegraphics[width=0.3\textwidth]{etsiit_logo.png}\\[0.1cm]
\vspace{1.5cm}
\includegraphics[width=0.5\textwidth]{decsai.jpg}\\[0.1cm]
\vspace{1cm}
\textsc{Escuela Técnica Superior de Ingenierías Informática y de Telecomunicación}\\
\vspace{1cm}
\textsc{Curso 2017-2018}
\end{minipage}
\end{titlepage}
\tableofcontents
\listoffigures
\thispagestyle{empty}
\newpage
\setcounter{page}{1}

\section{Enunciado}

\textbf{Ejercicio}. Resuelva las siguientes ecuaciones de recurrencia mediante el método de \textit{expansión}.

\begin{equation}
T(n)=2T(\frac{n}{2})+1 \quad \text{con T(1)=1}
\end{equation}
\begin{equation}
T(n)=2T(\frac{n}{2})+n \quad \text{con T(1)=1}
\end{equation}

\section{Resolución}
El método de expansión consiste en ir resolviendo la ecuación de la forma $T(n)$ hasta que llegue un punto en el que podamos obtener un patrón de resolución del que hallaremos una ecuación.
En ese momento, aplicaremos la ecuación al caso base y acotaremos mediante la notación \textbf{O(n)}.
\subsection{Ecuación 1}
Para este ejercicio iremos expandiendo en forma $T(\frac{n}{2^i})$.
\[T(n)=2T(\frac{n}{2})+1 \quad \text{con T(1)=1}\]
Comenzamos desarrollando la ecuación para $i=1 (\frac{n}{2})$
\[T(n)= 2T(\frac{n}{2})+1 =2 \cdot \left[2T(\frac{n}{4})+1\right]+1 = 4T(\frac{n}{4})+3= \]
Seguimos ahora con $i=2 (\frac{n}{4})$
\[= 4 \cdot \left[2T(\frac{n}{8})+1\right]+3 = 8T(\frac{n}{8}) + 7 = \]
Intuimos el patrón que sigue la recurrencia:
\[T(\frac{n}{2^i})=
\begin{cases}
	1 & \quad \text{si } i = 0  \\
	2^{i+1} \cdot T(\frac{n}{2^{i+1}}) + 2^{i+1} - 1 & \quad \text{en otro caso}
\end{cases}
\]
Lo comprobamos para $i=3$
\begin{enumerate}
\item Por expansión
\[ = 8 \cdot \left[2T(\frac{n}{16})+1\right]+7=16T(\frac{n}{16})+15\]
\item Mediante la ecuación del patrón
\[ =2^{3+1} \cdot T(\frac{n}{2^{3+1}}) + 2^{3+1} - 1 = 16T(\frac{n}{16})+15 \]
\end{enumerate}
Por tanto, nuestro patrón era acertado. Ahora, aplicamos al \textit{caso base} ($\frac{n}{2^{i+1}} = 1 \rightarrow n=2^{i+1} \rightarrow i=\log_2n-1$)

\begin{equation}
\begin{split}
&T(\frac{n}{2^{\log_2n}})=2^{\log_2n-1 + 1} \cdot T(1) + 2^{\log_2n-1+1} - 1= 2 \cdot 2^{\log_2n} + 1 = 2^{\log_2n+1} - 1 \simeq 2n-1
\end{split}
\end{equation}

\begin{center}
Por tanto, concluimos que $T(n)$ es del orden de $O(n)$ en la notación \textbf{O}.
\end{center}

\subsection{Ecuación 2}
\[T(n)=2T(\frac{n}{2})+n \quad \text{con T(1)=1}\]
Vemos que el ejercicio es prácticamente igual al anterior, por lo que sólo tendremos que añadir una $n$ multiplicando al último patrón. No obstante, desarrollamos para comprobarlo: 
Comenzamos desarrollando la ecuación para $i=1 (\frac{n}{2})$
\[T(n)= 2T(\frac{n}{2})+n =2 \cdot \left[2T(\frac{n}{4})+n\right]+1 = 4T(\frac{n}{4})+3n= \]
Procedemos con $i=2 (\frac{n}{4})$
\[= 4 \cdot \left[2T(\frac{n}{8})+n\right]+3 = 8T(\frac{n}{8}) + 7n = \]
Vemos que el patrón de la ecuación es probablemente el siguiente: 
\[T(\frac{n}{2^i})=
\begin{cases}
	1 & \quad \text{si } i = 0  \\
	2^{i+1} \cdot T(\frac{n}{2^{i+1}}) + (2^{i+1} - 1)n & \quad \text{en otro caso}
\end{cases}
\]
Comprobamos para $i=3$
\begin{enumerate}
\item Por expansión
\[ = 8 \cdot \left[2T(\frac{n}{16})+n\right]+7n=16T(\frac{n}{16})+15n\]
\item Mediante la ecuación del patrón
\[ =2^{3+1} \cdot T(\frac{n}{2^{3+1}}) + (2^{3+1} - 1)n = 16T(\frac{n}{16})+15n \]
\end{enumerate}
Por tanto, nuestro patrón es correcto.
Aplicamos al \textit{caso base} ($i=\log_2n-1$):
\begin{equation}
\begin{split}
& T(\frac{n}{2^{\log_2n}})=2^{\log_2n-1 + 1} \cdot T(1) + (2^{\log_2n-1+1} - 1)n=\\
& = 2^{\log_2n} + (2^{\log_2n} - 1)n \simeq 2n \cdot (2n-1)n = 4n^2-2n
\end{split}
\end{equation}
\begin{center}
Por tanto, podemos concluir que la ecuación $T(n)$ es del orden de $O(n^2)$ en notación \textbf{O}.
\end{center}
\end{document}